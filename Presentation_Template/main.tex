\documentclass{beamer} 
\usepackage[utf8]{inputenc}
\usepackage{colortbl}
\usepackage{tikz}
\usepackage{lmodern}
\usepackage{amsthm}
\usepackage{amsmath}
\usepackage[ruled,vlined]{algorithm2e}
\usepackage[noend]{algorithmic}
\usepackage{centernot} 
\usepackage{caption}
\algsetup{indent=2em} 
\renewcommand{\algorithmiccomment}[1]{\hspace{2em}// #1} 
\usepackage{listings}
\usepackage{epsfig, amssymb, geometry}
\usepackage{latexsym, graphics, graphicx}
\usepackage{amsxtra,mathtools}
\usepackage{textcomp}
\usepackage{mathptmx,mathrsfs, trajan}
\usepackage{float}
\usepackage{wasysym}
\usetikzlibrary{automata, positioning,arrows.meta,quotes}
\usepackage{comment}
\usepackage{lineno}
\usepackage{semantic}
\usepackage{enumitem}
\usepackage{mathtools}
\usepackage{ifthen}
\usepackage{xcolor}
\usepackage{proof}
\usepackage{changepage}
\usepackage{extarrows}
\usepackage{fancyvrb}
\usepackage{textpos}
\usepackage[font=small,labelfont=bf]{caption}
\usepackage{multicol}
\usepackage{qtree}
\usetikzlibrary{circuits.logic.US}
\usepackage{chronology}
\usetheme{Madrid}
\useoutertheme{infolines} 
\useinnertheme{circles}
\usetikzlibrary{arrows}
\usetikzlibrary{decorations.markings}
\usetikzlibrary{decorations.pathmorphing}
\setbeamertemplate{navigation symbols}{}
\newenvironment{myitemize}{\begin{itemize}\setlength\itemsep{3ex}}{\end{itemize}}
\newenvironment{myenumerate}{\begin{enumerate}\setlength\itemsep{3ex}}{\end{enumerate}}
\newcounter{result}[section]
\makeatletter
\setbeamertemplate{footline}
{
  \leavevmode%
  \hbox{%
  \begin{beamercolorbox}[wd=.7\paperwidth,ht=2.25ex,dp=1ex,center]{title in head/foot}%
    \usebeamerfont{title in head/foot}\insertsection
  \end{beamercolorbox}%
  \begin{beamercolorbox}[wd=.3\paperwidth,ht=2.25ex,dp=1ex,right]{date in head/foot}%
    \insertframenumber{} / \inserttotalframenumber\hspace*{2ex}
  \end{beamercolorbox}}%
  \vskip0pt%
}
\makeatother
\date{}%leave blank
\author[]{}%leave blank
\setbeamersize{text margin left=12mm,text margin right=12mm}

 %%%%%%%%%%%%%%%%%%%%%%%%%%%%%%%%%%%%%%%%%%%%%%%%%%%%
 %Start editing from here
 %%%%%%%%%%%%%%%%%%%%%%%%%%%%%%%%%%%%%%%%%%%%%%%%%%%%

\title[]{Compiler Construction \& Tarjan's Algorithm}
\subtitle{Understanding SCCs in Call Graph Analysis\newline \today}


\definecolor{bottomLeft}{RGB}{108, 166, 205}%color for bottom left footer, and Block top labels

\definecolor{headers}{RGB}{0, 191, 255}%color for headers on slides, and table of contents, itemize etc.

\definecolor{writing}{RGB}{255,255,255}%color for writing in headers section

\definecolor{mynavy}{RGB}{36,52,102}%color for text on slides

\setbeamercolor{palette primary}{bg=headers,fg=writing}
%bg is for headers, fg is for text
\setbeamercolor{palette secondary}{bg=bottomLeft,fg=writing}
%bg is for left half of footer, fg is for writing
\setbeamercolor{structure}{fg=headers} % itemize, enumerate, etc
\setbeamercolor{section in toc}{fg=headers} % Table of contents sections

%Change text color
\setbeamercolor {normal text} {fg=mynavy} 
\usebeamercolor* {normal text}

\setbeamercolor{block title}{bg=bottomLeft,fg=mynavy}

%below is the logo, you can leave it, remove or replace it
\titlegraphic{\includegraphics[width=5cm,height=1.5cm]{newpaltzlogo.jpg}}

%below are two options for separating sections
\AtBeginSection[]{

   % \begin{frame}[plain]
    %    \sectionpage
    %\end{frame}
     \begin{frame}
    \frametitle{Table of Contents}
    \tableofcontents[currentsection]
  \end{frame}
}
%starts the documents
\begin{document}
{
%this sets an image as a background for the first slide
\setbeamertemplate{background} 
{
    \includegraphics[width=\paperwidth,height=\paperheight]{science-mathematics-abstract-background-circles-cube-triangles-lot-lines-sacred-geometry-backdrop-79621056.jpg}
}
\frame{\titlepage}
}

%first section
\section{First Section}
%creates a frame
\begin{frame}{Frame Title}%Frame Title
This is a sample frame.

   %creates a list of things
   \begin{itemize}
       \item Item 1
       \item[-] Item 2 %you can choose the bullet point you want
   \end{itemize}

   \bigskip
   
%creates a list with larger gaps
\begin{myitemize}
    \item[$\bullet$] Item A
    \item[+] Item B
\end{myitemize}
\end{frame}
\begin{frame}{Frame Title}
%this is how you can add a definition
    \begin{definition}
        A \underline{\textbf{\emph{new word}}} is ...
       
    \end{definition}
    \pause% this is how you can create a slide that appears step-by-step

    %this is an example block
    \begin{block}{Block Title}
        Here is a really cool example. 
    \end{block}
\pause
\begin{block}{Block Title}
   Here is the answer to that really cool example
\end{block}

\end{frame}

\subsection{New SubSection}

\begin{frame}{Frame with photo}
\begin{figure}
    \centering
       \includegraphics[scale=0.2]{regLang.jpg}

\end{figure}
\end{frame}

\section{New Section}

\begin{frame}{Frame with multiple photos}
\centering
%only is how we can choose which slides something appears on.
%this picture will appear on slide 1
\only<1>{
    \includegraphics[]{motivation.png}}
    %this picture will appear on slide 2
    \only<2>{\includegraphics[scale=0.8]{motivation2.png}}
%this text will appear on slide 1,2,3
\only<1,2,3>{HELLO}
%this text will appear on slide 3
\only<3>{GOODBYE}
   
\end{frame}

\begin{frame}{References}
    \begin{thebibliography}{}
%a book
\bibitem{BN99}
Franz Baader, Tobias Nipkow:
Term Rewriting and All That.
Cambridge University Press, 1999.

%a book
\bibitem{Sipser}
Michael Sipser: Introduction to the Theory of Computation, Cengage, 119-120, 1997. 

%a paper
\bibitem{Zhang}
Jielan Zhang, Zhongsheng Qian: The Equivalent Conversion between Regular Grammar and Finite Automata, Journal of Software Engineering and Applications,~6(1), 2013.

%a website
\bibitem{website1}
\url{https://en.wikipedia.org/wiki/Computer_science} 
\end{thebibliography}
\end{frame}



{
\setbeamertemplate{background} 
{
    \includegraphics[width=\paperwidth,height=\paperheight]{science-mathematics-abstract-background-circles-cube-triangles-lot-lines-sacred-geometry-backdrop-79621056.jpg}
}

\begin{frame}{}
    \begin{huge}
        \begin{center}
        \textbf{ {Thank You! \\ Questions?}}\\
           
        \end{center}
    \end{huge}
\end{frame}
}


\end{document}